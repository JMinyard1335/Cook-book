% Created 2025-06-16 Mon 10:46
% Intended LaTeX compiler: pdflatex
\documentclass[11pt]{article}
\usepackage[utf8]{inputenc}
\usepackage[T1]{fontenc}
\usepackage{graphicx}
\usepackage{longtable}
\usepackage{wrapfig}
\usepackage{rotating}
\usepackage[normalem]{ulem}
\usepackage{amsmath}
\usepackage{amssymb}
\usepackage{capt-of}
\usepackage{hyperref}
\usepackage[margin=.75in]{geometry}
\usepackage{graphicx}
\setcounter{secnumdepth}{2}
\author{Jachin Minyard}
\date{\today}
\title{Cookbook}
\hypersetup{
 pdfauthor={Jachin Minyard},
 pdftitle={Cookbook},
 pdfkeywords={},
 pdfsubject={},
 pdfcreator={Emacs 30.1 (Org mode 9.7.11)}, 
 pdflang={English}}
\begin{document}

\maketitle

Each recipe by default makes what I would call 1 serving with one serving being enough for a family of 4 most of the time.
\section{Breads}
\label{sec:org6c9835c}
\subsection{Monkey Bread}
\label{sec:org6a017b2}
\subsection{Pumpkin Bread}
\label{sec:org371571b}
\subsubsection*{Ingredients}
\label{sec:org50e2db3}
\begin{itemize}
\item 3 Cups of sugar
\item 1 cup salad oil
\item 4 eggs beaten
\item 1 can Del Monte Pumpkin || 1 3/4 cups of Pumpkin puree
\item 3 1/2 Cups of flour
\item 2 tsp of baking soda
\item 2 tsp salt
\item 1 tsp baking powder
\item 1 tsp nutmeg
\item 1 tsp all spice
\item 1 tsp cinnamon
\item 1/2 tsp cloves
\item 2/3 cups of water
\end{itemize}
\subsubsection*{Instructions}
\label{sec:orgb780c37}
\begin{itemize}
\item Cream the sugar and oil together.
\item Add eggs and pumpkin, mix well
\item Shift Dry ingredients together
\item Add dry ingredients and water, alternating back and forth to avoid clumps.
\item Poor into 2 well greased floured 9x5 inch loaf pans
\item Bake at 350 \degree for 1 1/2 hrs or until done (check by poking with a tooth pick)
\item Let stand for 10 minutes
\item remove from pans to cool.
\end{itemize}
\subsection{Banana Bread}
\label{sec:org194c5f7}
\subsection{Homemade Pie Crust}
\label{sec:orgce50236}
\subsubsection*{Ingredients}
\label{sec:org5eb42db}
For a crust without a crust cover use the single crust ingredients if making a crust with a top use the Double Crust
\paragraph*{Singe crust}
\label{sec:orgb018a50}
\begin{itemize}
\item 1 3/4 cups of sifted all-purpose flour
\item 1/2 tsp salt
\item 2/3 cup of shortening
\item 4-5 Tbsp of cold water
\end{itemize}
\paragraph*{Double crust}
\label{sec:orgf797a98}
\begin{itemize}
\item 2 1/2 cups of sifted all-purpose flour
\item 1 tsp salt
\item 1 cup of shortening
\item 6 Tbsp of cold water
\end{itemize}
\subsubsection*{Instructions}
\label{sec:orgbe261bf}
\begin{itemize}
\item Sift the flower into a bowl and salt
\item Add half the shortening and mix
\item Add the rest of the shortening and finish mixing
\item add a little of the cold water at a time kind of poking with a fork
\item keep adding cold water till the dough is kinda flaky
\end{itemize}

The Key to making a crust is to not make an actual dough the crust needs to remain flaky.
This does make the crust harder to work with. but makes a better crust.
\section{Deserts}
\label{sec:org8449bcf}
\subsection{Apple Pie}
\label{sec:org0addaa9}
\subsubsection*{Ingredients}
\label{sec:org27ab885}
\begin{itemize}
\item 6->8 apples
\item 3 Tbsp of flour
\item 1/2 cup of sugar
\item 1 Tbsp of cinnamon
\end{itemize}
\subsubsection*{Instruction}
\label{sec:org9b24014}
\begin{itemize}
\item 

\item 

\item 
\end{itemize}
\section{Ethiopian}
\label{sec:org7be566f}
\subsection{Misir Wot}
\label{sec:org4db90c9}
\subsubsection*{Ingredients}
\label{sec:orgbdcf25a}
\begin{itemize}
\item 4 tbsp Spoons of Niter Kibbeh
\item 1 Large Yellow Onion, Diced
\item 3 cloves of garlic, Minced
\item 1 Roma Tomato, Finely Chopped
\item 3 tbsp of Tomato Paste
\item 2 tbsp Spoons of Berbere
\item 1 Cup of Red Lentils
\item 2 1/2 Cups of Broth (chicken or vegetable)
\item 1 tsp of salt
\end{itemize}
\subsubsection*{Instructions}
\label{sec:org84bfbf3}
\begin{enumerate}
\item Melt 3 tbsp of Niter Kibbeh in a large pot over medium heat.
\item Add onions and cook for around 8-10 mins or until golden brown
\item Add garlic, tomatoes, tomato paste, and 1 tbsp of Berbere. Cook for 5-8 mins or until the tomatoes have cooked down a bit.
\item Add Lentils and broth. Bring to a boil and then reduce to a simmer. Cook for 30-40 mins or until the lentils are soft. (stir occasionally)
\item Stir in the rest of the niter kibbeh and berbere. Add salt to taste.
\end{enumerate}
\subsection{Shiro Powder}
\label{sec:orgf161d21}
\subsubsection*{Ingredients}
\label{sec:org1a18968}
\begin{itemize}
\item 3/4 Cup of Chickpea Flour
\item 1 tbsp berbere
\item 1 tsp ground cardamom
\item 1 tsp cumin powder
\item 1 tsp garlic powder
\item 1tsp salt
\end{itemize}
\subsubsection*{Instructions}
\label{sec:orgb32bec4}
\begin{enumerate}
\item on skillet dry roast the ingredients
\end{enumerate}
\subsection{Shiro Wot}
\label{sec:org33c1452}
\subsubsection*{Ingredients}
\label{sec:org4fd7b95}
\begin{itemize}
\item 1 cup Shiro Powder
\item 3 tbsp of Niter Kibbeh
\item 1 large onion, diced
\item 1 tsp of garlic, minced
\item 2 roma tomatoes, finely chopped
\end{itemize}
\subsubsection*{Instructions}
\label{sec:orgdddf6e3}
\begin{enumerate}
\item Add the Niter Kibbeh to a pot and melt over medium heat.
\item Add the onions and cook for 8-10 mins till caramelized.
\item Add the tomatoes and garlic. Cook till reduced stirring occasionally.
\item Add the Shiro Powder and stir till there are no more dry lumps.
\item Add 2 cups of water and bring to a boil. Reduce to a simmer and cook for 20-30 mins.
\end{enumerate}
\subsection{Tikil Gomen}
\label{sec:org5522d33}
\subsubsection*{Ingredients}
\label{sec:org23d2353}
\begin{itemize}
\item 1 Cabbage diced
\item 1 Yellow Onion sliced long ways
\item 4 Large Carrots sliced in to slivers
\item 4 Green Chilies chopped
\item 4 Cloves of Garlic minced
\item 2 Spoons of Turmeric
\item 1 Spoon of Bebere
\item 3 tbsp of Niter Kiber
\end{itemize}
\subsubsection*{Instructions}
\label{sec:org2946e69}
\begin{enumerate}
\item Add Niter Kiber to a pan, melt over medium heat.
\item Add onions and cook till clear
\item Add Carrots, garlic, and peppers cook till carrots start to soften.
\item Add bebere and turmeric mix well.
\item Add the cabage and cover. cook for 10 mins stirring occasionally
\end{enumerate}
\subsection{Ye'abasha Gomen}
\label{sec:orgb106826}
\subsubsection*{Ingredients}
\label{sec:org623c8eb}
\begin{itemize}
\item 10 ounces of collard greens, chopped
\item 3 tbsp of Niter Kibbeh
\item 1 1/2 tsp of ginger, minced
\item 2 tsp of garlic, minced
\item 1 large white onion diced
\item 1 tsp of smoked paprika
\item 1/2 tsp of cardamon
\item 1 tsp of coriander/cumin
\item 1-2 fresh chili peppers, minced
\item 2 tbsp of lemon juice
\end{itemize}
\subsubsection*{Instructions}
\label{sec:org9e374a3}
\begin{enumerate}
\item Add Niter Kibbeh, garlic, ginger, peppers, and other dried spices to a large skillet and saute for 30 sec. Don't burn them!
\item Add the onions and mix well. Saute for another 5 min.
\item Toss in the collared greens and lemon juice. Turn down heat and cook till collared greens are wilted.
\end{enumerate}
\subsection{Fasollia}
\label{sec:orgda779da}
\subsubsection*{Ingredients}
\label{sec:orgdb8b584}
\begin{itemize}
\item 1lb of green beans, chopped (remove the ends)
\item 1lb of carrots, julienned (cut longways into strips)
\item 1 medium onion, diced
\item 1 tbsp of garlic, minced
\item 1 tsp of ginger, grated
\item 2 tbsp of Niter Kibbeh
\item salt as needed
\item 1/4 cup of water (more as needed).
\end{itemize}
\subsubsection*{Instructions}
\label{sec:org55814d4}
\begin{enumerate}
\item Add the green beans to a pan and cook till they start to brown and reduce. add a little salt, The idea is to dry them out as much as possible.
\item Take the green beans out and set aside.
\item Add the niter kibbeh to the pan and add onions. Cook for about 5 mins.
\item Add the garlic and ginger. Cook for another 2 mins
\item Add the carrots and water and simmer for about 10 mins. Stir occasionally, add more water as needed.
\item Add the green beans back and simmer over medium heat for another 10 mins.
\end{enumerate}
\subsection{Niter Kibbeh}
\label{sec:org8f22cec}
\subsubsection*{Ingredients}
\label{sec:orgc1ca055}
\begin{itemize}
\item 1 lb unsalted butter
\item 1/4 yellow onion, minced
\item 3 tbsp of garlic, minced
\item 2 tbsp of ginger, minced
\item 1 2in cinnamon stick
\item 1 tsp black peppercorn, whole
\item 3 black cardamom pods, whole
\item 1 tsp fenugreek seeds
\item 1 tsp coriander seeds
\item 1 tsp dried oregano
\item 1/2 tsp cumin seeds
\item 1/4 tsp ground nutmeg
\item 1/4 tsp ground turmeric
\item 1 tbsp of Beso Bila (Ethiopian basil)
\item 1 tbsp of Kosseret (Ethiopian herb)
\end{itemize}
\subsubsection*{Instructions}
\label{sec:orga9a25d8}
\begin{enumerate}
\item place dry herbs on a skillet and roast till fragment. Careful not to scorch the spices.
\item Place all ingredients in a saucepan bring to a low simmer and cook for 60-90 mins. Careful not to burn the butter or it will become bitter.
\item Pour through cheese cloth to strain out all the herbs and spices. Place in an airtight jar and store it.
\end{enumerate}
\section{Indian}
\label{sec:org88489fc}
\subsection{Masala paste}
\label{sec:orgf208416}
\subsubsection*{Ingredients}
\label{sec:org5a6d90e}
\begin{itemize}
\item 1 cup coconut, grated
\item 8 cashews
\item 1 in or ginger
\item 1 clove of garlic
\item 2 green chili's
\item 2 tbsp poppy seeds
\item 1 tsp coriander seeds
\item 1/2 tsp fennel
\item handful of coriander
\item 1/4 cup of water
\end{itemize}
\subsubsection*{Instructions}
\label{sec:org6cf52fc}
\begin{enumerate}
\item put ingredients and water in a food processor and blend till smooth. add water as needed.
\end{enumerate}
\subsection{Veggie Korma}
\label{sec:orgc71206b}
\subsubsection*{Ingredients}
\label{sec:orgf0a456c}
\begin{itemize}
\item 4 tsp of olive oil
\item 1 bay leaf
\item 1 2in cinnamon stick
\item 2 pod of cardamon or 1/2 tsp of cardamon seeds
\item 3 cloves
\item 1 yellow onion, finely chopped
\item 1 roma tomato, finely chopped
\item 1 carrot, chopped or handful of baby carrots, chopped
\item 
\end{itemize}
\subsubsection*{Instructions}
\label{sec:org93b1d13}
\begin{enumerate}
\item 

\item 

\item 

\item 

\item 

\item 

\item 
\end{enumerate}
\subsection{Paruppu Sadham}
\label{sec:org7ba80eb}
This is a classic rice and dal dish from Coimbatore. For a 2.0 ltr pressure cooker half the recipe.
\subsubsection*{Ingredients}
\label{sec:org7485571}
\begin{itemize}
\item 1 cup of rice.
\item 1/4 cup of Toor dal
\item 2 tbsp of cooking oil
\item 2 tbsp of Ghee
\item 1/2 tsp Dried Mustard
\item 1 tsp cumin
\item 6 - 8 Curry Leaves
\item 1 tsp Black pepper corns (whole)
\item 5 cloves of garlic (whole)
\item 1 - 2 Green Chilies
\item 1 - 2 Dried Red Chilies
\item 1/2 tsp hing
\item 1 small red onion
\item 2 Roma Tomato
\item 1/4 tsp turmeric powder
\item 1/2 tsp chili powder
\item hand full of fresh coriander
\item 3 cups of water
\item salt to taste.
\end{itemize}
\subsubsection*{Instructions}
\label{sec:org3d7125f}
\begin{enumerate}
\item Add the dal and rice to a bowl. Rinse until the water is clear and all the dirt has been removed.
\item After rinsing the rice and dal add just enough water to let it soak for 30min. (fill till water is just above the top of the rice)
\item Add oil and Ghee to the pressure cooker, followed by the mustard, cumin, black pepper, garlic cloves, and chilies. Saute for 1 min.
\item Add the Hing and onions and curry leaves. Saute for another min.
\item Add the Tomato's, Turmeric, and chili powder, and salt to taste. Cook till the tomato's become soft.
\item Add the soaked dal, rice, and water.
\item Cover with pressure cooker lid and cook for 3 Whistles.
\item Take off heat let pressure vent.
\item Remove lid add coriander and mix well.
\end{enumerate}
\subsection{Andhra Chicken Curry}
\label{sec:orgb531969}
\subsubsection*{Ingredients}
\label{sec:orgc34771e}
\begin{itemize}
\item 2 tbsp Oil
\item 1 red onion
\item 3 curry peppers
\item 1 Roma Tomato
\item 6 chicken tenderloins
\item 1 tbsp Ginger Garlic paste
\item 1 cup of water
\item 4 tsp of salt
\item chili powder
\item turmeric
\item cumin
\item coriander
\item Gram Masala
\item Chicken Masala
\end{itemize}
\subsubsection*{Instructions}
\label{sec:orgcbda0ae}
\begin{enumerate}
\item Add oil to a pressure pot
\item Add the onion and a little bit of salt.
\item Add the curry Peppers
\item Add the Ginger Garlic Paste and mix well.
\item Add Chopped Chicken to the pot. add another 2tsp of salt
\item Add Spices.
\item Add tomato and mix
\item Bring to boil and then pressurize. Cook for 10  mins, or till chicken is cooked.
\end{enumerate}
\subsection{Bhindi Fry}
\label{sec:orgf3fed58}
\subsubsection*{Ingredients}
\label{sec:org341a5f3}
\begin{enumerate}
\item 1lb of Bhindi.
\item 1/4 cup of gram flour
\item 1/4 cup of rice flour
\item 1 Tbsp of corn flour
\item 1/4 Tsp of turmeric powder
\item 2 Tsp of Chili powder
\item 1 Tsp of Coriander powder
\item 1 Tsp of Cumin powder
\item 1 Tsp of salt
\item Canola Oil (Frying)
\item Olive oil (Roasting)
\item Cashews
\item 4-6 Green Chili's
\item Curry Leaves
\item Peanuts (optional)
\end{enumerate}
\subsubsection*{Instructions}
\label{sec:org14d0a40}
\begin{enumerate}
\item Mix the Flour's and Spices together in a bowl make sure you mix it well.
\begin{enumerate}
\item 1/4 cup of gram flour
\item 1/4 cup of rice flour
\item 1 Tbsp of corn flour
\item 1/4 Tsp of turmeric powder
\item 2 Tsp of Chili powder
\item 1 Tsp of Coriander powder
\item 1 Tsp of Cumin powder
\item 1 Tsp of salt
\end{enumerate}
\item Wash Dry and chop up the Bhindi in to pieces.
\item Mix the flour and spices with the Bhindi pieces.
\item Sprinkle on a little water to help it bind, let sit for 3-5min.
\item Add fry oil to large pan(don't fill the pan shallow fry)
\item add Bhindi cook and set aside
\item Roast:
\begin{enumerate}
\item Roast the curry leaves, set aside
\item Roast the cashews till golden brown, set aside
\item Roast Peanuts, set asside
\item Slit and roast the peppers, set aside
\end{enumerate}
\item Add the roasted ingredients to the fried Bhindi and mix well.
\end{enumerate}
\subsection{Sambar Masala}
\label{sec:orgfd5b9a7}
\subsubsection*{Ingredients}
\label{sec:org6438676}
\subsubsection*{Instructions}
\label{sec:orgc3501eb}
\subsection{Sambar}
\label{sec:orgaee17e6}
\subsubsection*{Ingredients}
\label{sec:org4d81e25}
\subsubsection*{Instructions}
\label{sec:org4e037ac}
\section{Thai}
\label{sec:org425bbd1}
\section{Mexican}
\label{sec:org99a29f5}
\section{Italian}
\label{sec:org16dc2bb}
\section{Spice Mixes}
\label{sec:org0a9e4de}
\section{Hungarian}
\label{sec:orgb4a07b7}
\subsection{Hungarian Caramelized Cabbage and Noodles}
\label{sec:org9e397e9}
\subsubsection*{Ingrediants}
\label{sec:org6fb5b22}
\begin{enumerate}
\item 2.5 lbs cabbage, finely shreaded (about 10 cups/2.5L, loosely packed)
\item 1 tsp salt
\item .5 cup olive oil
\item 1.5 tsp black pepper
\item 8 oz noodles
\end{enumerate}
\subsubsection*{Instructions}
\label{sec:org411e264}
\begin{enumerate}
\item In a large colander, mix cabbage and salt.  Set aside for 30 to 45
minutes.  Using your hands, squeeze excess liquid from cabbage and
transfer to a bowl.
\item In a large pot, over medium head, add oil to a hot pan.  Add
cabbage and stir to coat.  Cook, stirring occasionally, for about 5
minutes, until wilted.  Cover, reduce heat to low, and cook,
stirring occasionally, for about 1 hour, until tender and golden.
\item Increase heat to medium and cook, uncovered, stirring often and
scraping up browned bits from the bottom of the pan, for about 12
to 20 minutes, until cabbage is brown and very soft.
\item Stir in 1tsp pepper.
\item Meanwhile, cook noodles until tender but firm and drain.
\item Add noodles to cabbage and mix.  Add more pepper to taste.
\end{enumerate}
\end{document}
